\documentclass[12pt]{article}
\usepackage[margin=1in]{geometry}

\usepackage{amsmath,amssymb}
\usepackage{graphicx}
\usepackage{setspace}

\usepackage{xcolor,hyperref}

\definecolor{UMBlue}{RGB}{0 39 76}

\hypersetup{
  colorlinks,
  allcolors=UMBlue
%    citecolor=black,
%    filecolor=black,
%    linkcolor=black,
%    urlcolor=black
}



%\setcounter{figure}{0}
%\renewcommand{\thefigure}{\arabic{figure}}

%\usepackage{helvet}
%\renewcommand{\familydefault}{\sfdefault}

\title{Data Science Companion}

\author{Greg Simon, \url{gregorygsimon@gmail.com}}


\begin{document}

\maketitle


\abstract{A reference for basic data science tools and vocabulary, explaining
  essential terms and concepts, examining core ideas in major areas, and putting
  methods in context. Includes relevent keywords and references for further}

\tableofcontents
\newpage

\section{Natural Language Processing}

\section{Deep Learning}

\section{Time series \& Forecasting}

\subsection{ARIMA}

An $\text{ARIMA}(p,d,q)$ is an {\sl autoregressive integrated moving-average}
with $p$ autoregressive terms (AR), $d$ differencings, and $q$ moving average (MA) terms.

\begin{align*}
  \phi(B) (1-B)^d Y_t = c + \theta(B) \epsilon_t
\end{align*}
where
\begin{itemize}
\item $B$ is the back-shift/lag operator $BY_t = Y_{t-1}$.
\item $\phi(B) = (1 - \phi_1 B - \ldots - \phi_p B^p )$ is the autoregressive
  $\text{AR}(p)$ component
\item $c$ is a constant
\item $\theta(B) =  1 + \theta_1 B + \ldots \theta_q B^q$ is the moving
    average of the errors $\text{MA}(q)$ component.
\item $\epsilon_t$ is the error of the $\text{AR}(p)$ model at time $t$
\item The $(1-B)^d$ term induces $d$ differencing
\end{itemize}

%\subsubsection{Typical workflow}

% ACF diagrams
% testing for stationarity
% differencing for stationarity


\subsection{In {\tt R}}

\begin{itemize}
\item[{\tt auto.arima}] utilizes AIC and MLE to decide on best ARIMA parameters
\end{itemize}

\subsection{In {\tt Python}}

\subsection{References}

\url{https://otexts.com/fpp2/non-seasonal-arima.html}


\end{document}