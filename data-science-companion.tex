\documentclass[12pt]{article}
\usepackage[margin=1in]{geometry}

\usepackage{amsmath,amssymb}
\usepackage{graphicx}
\usepackage{setspace}

\usepackage{xcolor,hyperref}

\definecolor{UMBlue}{RGB}{0 39 76}

\hypersetup{
  colorlinks,
  allcolors=UMBlue
%    citecolor=black,
%    filecolor=black,
%    linkcolor=black,
%    urlcolor=black
}


\usepackage[style=authoryear,
%bibstyle=authoryear,
citestyle=alphabetic,
maxcitenames=2,
%maxbibnames=5,
%natbib=true,
hyperref=true,
abbreviate=true,
doi=false,
isbn=false,
backend=bibtex]{biblatex}

\setlength{\bibitemsep}{0.5\baselineskip}

\newcommand{\mystrut}{\ensuremath{\vphantom{\sum}}}


\addbibresource{ds-bib.bib}





% \setcounter{figure}{0}

%\renewcommand{\thefigure}{\arabic{figure}}

%\usepackage{helvet}
%\renewcommand{\familydefault}{\sfdefault}

%%%% greg's defs

\newcommand{\EE}{\ensuremath{\mathbb{E}}}
\newcommand{\PP}{\ensuremath{\mathbb{P}}}



%%%%

\title{Data Science Companion}

\author{Greg Simon, \url{gregorygsimon@gmail.com}}


\begin{document}

\maketitle


\abstract{A reference for basic data science tools and vocabulary, explaining
  essential terms and concepts, examining core ideas in major areas, and putting
  methods in context. Includes relevent keywords and references.}

\tableofcontents
\newpage



\section{Bayesian Statistics}
\newrefsection

We follow \cite{Fonnesbeck2019}. Frequentist hypothesis testing has some flaws.
\begin{itemize}
\item We decide on a statistical test, null hypothesis, and significance level
  subjectively, often based on tradition
\item The null hypothesis is often clearly false with enough data.
\item Easy to misinterpret the results, e.g. p-values.
\end{itemize}

Instead of testing, we want {\bf estimation} to measure {\sl how different} two
groups are and we want to include a an estimate of the {\sl uncertainty}, both
due to our lack of knowledge of model parameters ({\sl 'epistemic uncertainty'})
as well as uncertainty due to stochasticity of the system ({\sl aleatory
  uncertainty}). 

The Bayesian approach stems from the equation:
\begin{align*}
  \underbrace{\mystrut\PP(\theta|y)}_{\text{posterior}} =
  \frac{\overbrace{\mystrut\PP(y|\theta)}^{\text{likelihood}}\cdot\,\overbrace{\mystrut\PP(\theta)}^{\text{prior}}}
  {\underbrace{\mystrut\PP(y)}_{\text{marginal likelihood}}}
\end{align*}
By inputing in prior probability distributions, we get posterior distributions
out. The general steps are as follows:
\begin{enumerate}
\item {\bf Specify a probability model} -- assign distributions for unknown parameters,
  data, covariates, etc.
\item {\bf Calculate the posterior distributions} -- difficult but there are many
  different methods for this.
\item {\bf Check your model} -- does it fit the data? Are the conclusions reasonable? Are the outputs sensitive to changes in the model structure?
\item {\bf Use the posterior distribution to get what you want} -- point estimates,
  credible intervales, quantiles, predictions, etc.
\end{enumerate}

One of the main challenges is calculating.

\subsection{Markov Chain Monte Carlo (MCMC)}


Recall $\PP(\theta|y) \propto \PP(y|\theta) \PP(\theta)$ and we can calculate the
right side. We don't know the normalizing constant $\PP(y) = \int_\theta P(y|\theta)
\PP(\theta) \, d\theta$.

Good explanation from \cite{12657}
\begin{quote}
  The goal of MCMC is to draw samples from the probability distribution on the
right without having to know its exact height at any point. The way MCMC achieves this is to
"wander around" on that distribution in such a way that the amount of time spent
in each location is proportional to the height of the distribution.

The simplest variant of the Metropolis-Hastings algorithm (independence chain sampling) achieves this as follows: assume that in every (discrete) time-step, we pick a random new "proposed" location (selected uniformly across the entire surface). If the proposed location is higher than where we're standing now, move to it. If the proposed location is lower, then move to the new location with probability p, where p is the ratio of the height of that point to the height of the current location. (i.e., flip a coin with a probability p of getting heads; if it comes up heads, move to the new location; if it comes up tails, stay where we are). Keep a list of the locations you've been at on every time step, and that list will (asyptotically) have the right proportion of time spent in each part of the surface. (And for the A and B hills described above, you'll end up with twice the probability of moving from B to A as you have of moving from A to B).

There are more complicated schemes for proposing new locations and the rules for
accepting them, but the basic idea is still: (1) pick a new "proposed" location;
(2) figure out how much higher or lower that location is compared to your
current location; (3) probabilistically stay put or move to that location in a
way that respects the overall goal of spending time proportional to height of
the location.
\end{quote}

Good breakdown of MCMC in PyMC3  and of a simple change point model is in \cite[Ch. 1]{pilon16_bayes}.

\subsubsection{Metropolis-Hastings algorithm}

\subsubsection{Gibbs sampling}

\subsubsection{Hamiltonian Monte Carlo}

Random-walk sampling (like Metropolis-Hastings) is not efficient in high
dimension. We want an algorithm that samples from the area of the
parameter space that contains most of the non-zero probability. This region is
called the {\sl typical set}.

The Hamiltonian Monte Carlo (HMC) avoids random walk behavior by simulating a
physical system governed by Hamiltonian dynamics. The math is explained well in
the following report \cite{Neal93probabilisticinference}.

We start with a state $\chi = (s,\phi)$ with a position $s$ and velocity $\phi$.
As in physics, the joint probability is proportional to an invariant Hamiltonian
function:
\begin{align*}
  \PP((s,\phi)) \propto \exp( -\mathcal{H}(s,\phi))
\end{align*}
where the Hamiltonian is the sum of the two types of energy:
\begin{align*}
  \mathcal H(s,\phi) = \underbrace{\mystrut E(s)}_{\text{potential}} +
  \underbrace{\mystrut K(\theta)}_{\text{kinetic}} = E(s) + \frac{1}{2} \sum_i \phi_i^2.
\end{align*}

These satisfy the differential equations:

\begin{align*}\frac{ds_i}{dt} &= \frac{\partial \mathcal{H}}{\partial \phi_i} = \phi_i \\
\frac{d\phi_i}{dt} &= - \frac{\partial \mathcal{H}}{\partial s_i}
= - \frac{\partial E}{\partial s_i}
\end{align*}

Often we assume that $\PP(\phi)$ is often taken to be the univariate Gaussian. So
we can use a {\bf leap-frog} time discretization of this differential equation.
Care must be maintained to preserve {\sl volume conservation} and {\sl time
  reversibility}. This performs half-step updates to the velocity at time
$t+\epsilon/2$ which is used to compute $s(t+\epsilon)$ and $\phi(t+\epsilon)$.
We also introduce an accept/reject stage with some probability to correct for
the bias of discretization as well as floating-point rounding errors.

\subsubsection{No-U-Turn Sampler (NUTS)}

Summarized as ``Adaptively Setting Path Lengths in Hamiltonian Monte Carlo''
\cite{DBLP:journals/jmlr/HoffmanG14}. Auto-tunes number of steps $L$ and step
size $\epsilon$. Uses a recursive algorithm to build a set
of likely candidate points, stopping automatically if it begins to retrace its
steps.

\subsection{Model Checking}

Two types: {\bf Convergence diagnostics} and {\bf Goodness of fit}. The former
is intended to detect lack of convergence in MCMC sample - e.g. to ensure you
have not halted your sampling too early.

A converged model is not guaranteed to be a good model. Goodness of fit is used
to check the internal validity of the odel by comparing predictions from the
model to the data used to fit the model.

Informally, you can plot and inspect the traces and histograms of the observed
MCMC sample (as in {\tt arviz.plot\_trace()}). The trace may clearly not yet be
convergent, or it may appear to be stationary about its mean.

Another informal method is to initialize with differenting starting values. If
each trace converges towards the same equilibrium, this is evidence for {\sl
  ergodicity}\footnote{ergodicity is the tendency for some Markov chains to converge to the true
unknown value from diverse starting states}. Carefull for {\sl metastability} --
characterized by stability for a long duration but then moving to another region
of the parameter space.

\subsubsection{Gelman-Rubin diagnostic}

Gelman-Rubin diagnostic uses multiple chains to check for convergence, based on
the idea that if multiple chains have converged then they should appear very
similar to one-another. It uses an analysis of variance approach, calculating
between-chain variance $B$ and a within-chain variance $W$ and assess whether
they are different enough to worry about lack of convergence.

This allows us to estimate the {\sl marginal posterior variance} of the
parameter of interest and defines an $\hat R$ statistics that should be close to 1 if the chains are
convergent.

\begin{align*}
    \widehat{ \text{Var}}(\theta | y) &= \frac{n-1}{n} W + \frac{1}{n} B\\[2ex]
  \hat{R} &= \sqrt{\frac{\widehat{\text{Var}}(\theta | y)}{W}}
\end{align*}



\bigskip

A great Q\&A and common misunderstandings for MCMC is summarized in \cite{robert2020markov}.

\printbibliography[heading=subbibnumbered]
\pagebreak

\section{General Machine Learning Concepts}

\newrefsection

\subsection{Model Selection}

For a model $S$, the {\sl prediction risk} is defined to be 

\begin{align*}
  R(S) = \sum_{i=1}^n \EE[ (\hat Y_i(S) - Y_i*)]
\end{align*}
where $\hat Y_i$ are the predicted values and $Y_i*$ are the values of a future
observation at covariate values $X_i$. The {\sl training error} is
\begin{align*}
  \hat R_{tr}(S) = \sum_{i=1}^n \left(\hat Y_i(S) - Y_i \right)^2
\end{align*}
The training error is a downward biased estimator for the prediction risk, and
\begin{align*}
  \text{optimism}(S) = \text{bias}\left( \hat R_{tr}(S) \right) = \EE(\hat R_{tr}(S)) - R(S) = -2 \sum_{i=1}^n \text{Cov}(\hat Y_i, Y_i)
\end{align*}

This leads to {\bf Mallow's $C_p$ statistic} defined by:
\begin{align*}
  \hat R(S) = \hat R_{tr}(S) + 2 d \sigma_\epsilon^2
\end{align*}
where $\sigma_\epsilon^2$ is the estimate of the standard deviation of the
models error and $d$ is the number of free inputs or basis functions in the model. 
\cite[\S7.4]{esl}.




\subsubsection{Akaike information criterion (AIC)}

The {\sl Akaike Information Criterion} is (proportional to)

\begin{align*}
  \text{loglik}_{\text{MLE}} - |S|
\end{align*}
The log-likelihood of the model at the MLE minus the dimension of free
parameters in the model.

We start with a set of models $\{M_1, M_2  \ldots \}$. Let $\widehat{f_j}(x) :=
\hat f(x , \widehat{\beta_j})$ be
the estimated probability function obtained by using the parameters $\beta_j$
that realize the maximimum likelihood estimate for model $M_j$.

One approach \cite[\S13.9]{wasserman} to consider the {\sl Kullback-Leibler
  distance} defined by:
\begin{align*}
  D(f,g) = \sum_x f(x) \log\left(\frac{f(x)}{\hat f(x)} \right)
\end{align*}
And specifically consider the risk function $R(f,\hat f) = \mathbb{E}(D(f,\hat
f))$. We can write
\begin{align*}
  D(f,g ) =sum_x f(x) \log f(x)  - \sum_x f(x)\log \hat f(x)
\end{align*}
So finding $\hat f$ that minimizes risk is equivalent to minimizing $a(f, \hat
f):= \mathbb{E}\left( \sum_x f(x) \log \hat f(x) \right)$. The AIC is an
appoximately unbiased estimate of $a(f,\hat f)$.

\subsubsection{Bayes information criterion (BIC)}

With models with a set of models $M_1,M_2, \ldots $ and observed data $Z$, we
have

\begin{align*}
\frac{ \PP (M_m | Z) }{\PP(M_\ell|Z)} = \frac{\PP(M_m)}{\PP(M_\ell)} \cdot \frac{\PP(Z | M_m)}{\PP(Z | M_\ell)} 
\end{align*}
The rightmost factor is what we are concerned with, the {\sl Bayes Factor}:
\begin{align*}
  \text{BF}(Z) = \frac{\PP(Z | M_m)}{\PP(Z | M_\ell)} 
\end{align*}
We seek to approximate the integral $\PP(Z | M_m) = \displaystyle\int \PP(Z | \theta,M_m)
\PP(\theta |M_m) d\theta$, (where the integral is over the space of parameters
$\theta$). This can be approximated using the MLE $\hat \theta_m$ for $M_m$
\begin{align*}
  \log \PP(Z |M_m) \approx \log \PP(Z | \hat \theta_m, M_m ) - \frac{d_m}{2} \log N + O(1)
\end{align*}
So if we take our loss function to be $-2 \log \PP(Z | \hat \theta_m , M_m)$
then we reach the BIC criterion with the goal of minimizing:
\begin{align*}
  \text{BIC} = -2 \text{loglik} + (\log N )\cdot d
\end{align*}

Another great resource for BIC is \cite{Raftery}.

\subsection{Fischer information}


\subsection{VC dimension}

In a simple case, consider two class classification problem where $f(\mathbf
x,\alpha) \in \{-1,1\}$ for some parameter $\alpha$. A given set of $\ell$
points can be labeled in $2^\ell$ possible ways. If there is a fucntion
$f(\cdot,\alpha)$ which correctly classifies all $\ell$ points, then we say
these points is {\sl shattered} by the set $\{f(\cdot,\alpha)\}$. The VC
dimension is the maximum number of training points that can be shattered by
$\{f(\alpha)\}$. If the VC dimension is $h$, then there is at least one set of
points $h$ that can be shattered but not {\sl every} set of $h$ points will be
shattered.  \cite[\S2.1]{Burges2004}



\subsection{Ensemble methods}

\subsubsection{Bagging}

\subsubsection{Boosting}


\printbibliography[heading=subbibnumbered]
\pagebreak

\section{Regression}

\subsection{Regularization}
\subsubsection{Ridge $(L^2)$}

\begin{align*}
  SSE_{L2} = \sum_i (y_i - \hat y_i)^2 + \lambda\sum_{j=1}^P \beta_j^2
\end{align*}

\subsubsection{Lasso $(L^1)$}

\begin{align*}
  SSE_{L1} = \sum_i (y_i - \hat y_i)^2 + \lambda\sum_{j=1}^P |\beta_j|
\end{align*}
Tends to shrink coefficients to 0 and almost performs feature selection. 

\subsection{Random Forests}

Data is split and the average

\subsection{Gradient Boosted Trees}

Pseudocode for GBM \cite{CASI,}
\begin{enumerate}
\item Select tree depth $D$ and number of iterations $K$
\item Compute average response $\bar y$ and use this as initial predicted value.
\item for $i = 1$ to $K$:
  \begin{enumerate}
  \item Compute the residual (observed - prediction), for each sample.
  \item Fit a regression tree of depth $D$ using residuals as the response.
  \item Predict each sample using the regression tree fit in the previous step.
  \item Updated predicted value of to the predicted value generated in previous step.
  \end{enumerate}
\end{enumerate}



\subsection{Bayesian Additive Regression Trees (BART)}


\section{Classification}
\newrefsection

\subsection{Metrics}

\subsubsection{Area under ROC curve}

Plot the True Positive Rate (aka Recall) vs False Positive Rate.

AUC is scale-invariant. It measures how well predictions are ranked, rather than
their absolute values. AUC is classification-threshold-invariant. It measures the quality of the model's predictions irrespective of what classification threshold is chosen.

Classification-threshold invariance is not always desirable. In cases where
there are wide disparities in the cost of false negatives vs. false positives,
it may be critical to minimize one type of classification error. For example,
when doing email spam detection, you likely want to prioritize minimizing false
positives (even if that results in a significant increase of false negatives).
AUC isn't a useful metric for this type of optimization.

\subsubsection{Mathews Correlation Coefficient}

The Pearson product moment correlation coefficient between actual and predicted
values \cite[Methods]{ChiccoJurman2020}. As a function of the Confusion Matrix, entries, this is:
\begin{align*}
  MCC = \frac{TP \cdot TN - FP \cdot FN}{\sqrt{(TP+FP)\cdot(TP + FN) \cdot (TN+FP) \cdot (TN+FN)})}
\end{align*}



\subsection{Support Vector Machines (SVM)}

\cite{Burges2004}

\subsection{Naive Bayes Classifying }


\printbibliography[heading=subbibnumbered]

%\printbibliography[heading=subbibnumbered]
\pagebreak

\section{Unsupervised Learning}
\newrefsection

%\printbibliography[heading=subbibnumbered]
\pagebreak

\section{Feature Engineering}
\newrefsection

\subsection{Principal Component Analysis (PCA)}



%\printbibliography[heading=subbibnumbered]
\pagebreak


%\section{Deep Learning}
%\newrefsection
%\printbibliography[title={References (Deep Learning)}]
%\pagebreak

\section{Natural Language Processing}
\newrefsection

\section{Embeddings}

\subsection{TF-IDF}


%\printbibliography[title={References (Deep Learning)}]
\pagebreak


\section{Time series \& Forecasting}
\newrefsection

\subsection{ARIMA}

An $\text{ARIMA}(p,d,q)$ is an {\sl autoregressive integrated moving-average}
with $p$ autoregressive terms (AR), $d$ differencings, and $q$ moving average
(MA) terms. \cite{fpp}.

\begin{align*}
  \phi(B) (1-B)^d Y_t = c + \theta(B) \epsilon_t
\end{align*}
where
\begin{itemize}
\item $B$ is the back-shift/lag operator $BY_t = Y_{t-1}$.
\item $\phi(B) = (1 - \phi_1 B - \ldots - \phi_p B^p )$ is the autoregressive
  $\text{AR}(p)$ component
\item $c$ is a constant
\item $\theta(B) =  1 + \theta_1 B + \ldots \theta_q B^q$ is the moving
    average of the errors $\text{MA}(q)$ component.
\item $\epsilon_t$ is the error of the $\text{AR}(p)$ model at time $t$
\item The $(1-B)^d$ term induces $d$ differencing
\end{itemize}

%\subsubsection{Typical workflow}

% ACF diagrams
% testing for stationarity
% differencing for stationarity


\subsection{In {\tt R}}

\begin{itemize}
\item[{\tt auto.arima}] utilizes AIC and MLE to decide on best ARIMA parameters
\end{itemize}

\subsection{In {\tt Python}}

\printbibliography[title={References (Time Series \& Forecasting)},heading=subbibnumbered]
\pagebreak


%\section{ExampleSectionName}
%\newrefsection
%\printbibliography[title={References (ExampleSectionName)},heading=subbibnumbered]
%\pagebreak



\end{document}